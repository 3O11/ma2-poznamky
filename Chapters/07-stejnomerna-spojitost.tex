\documentclass[../main.tex]{subfiles}

\begin{document}

%%%%%%%%%%%%%%%%%%%%%%%%%%%%%%%%%%%%%%%%%%%%%%%%%%%%%%%%%%%%%%%%%%%%%%%%%%%%%%%%%%%%%%%%%%%%%%%%%%%%%%%%%
\section{Stejnoměrná spojitost}
%%%%%%%%%%%%%%%%%%%%%%%%%%%%%%%%%%%%%%%%%%%%%%%%%%%%%%%%%%%%%%%%%%%%%%%%%%%%%%%%%%%%%%%%%%%%%%%%%%%%%%%%%
\begin{definition}[Stejnoměrná spojitost]
	Řekneme, že $f : (X,d) \rightarrow (Y,d')$ je stejnoměrně spojité, je-li
	\[\forall \varepsilon > 0 \exists \delta > 0 : \forall x,y : d(x,y) < \delta \implies d'(f(x),f(y)) < \varepsilon\]
\end{definition}

\begin{example}
	$f = (x \mapsto x^2) : \mathbb{R} \rightarrow \mathbb{R}$ je spojitá, ale ne stejnoměrně spojitá.
	Máme $|f(x) - f(y)| = |x+y| \cdot |x-y|$; tedy abychom dostali $|f(x)-f(y)| < \varepsilon $ v 
	blízkosti $x = 100$ potřebujeme $\delta$ stokrát menší než v blízkosti $x = 1$.
\end{example}

\begin{theorem}[Spojitost zobrazení na kompaktním prostoru]
	Je-li $(X,d)$ kompaktní, je každé spojité $f : (X,d) \rightarrow (Y,d')$ stejnoměrně spojité. Zejména to platí 
	pro spojité reálné funkce na kompaktních intervalech.
\end{theorem}

\begin{proof}
	Nechť $f : (X,d) \rightarrow (Y,d')$ není stejnoměrně spojité. Potom $\exists \varepsilon > 0 : \forall n\ \exists x_n, y_n :$
	\[d(x_n,y_n) < \frac{1}{n}\]
	ale
	\[d'(f(x_n),f(y_n)) \geq \varepsilon.\]
	Zvolme konvergentní podposloupnost $(x_{k_n})_n$ posloupnosti $(x_n)_n$. Označme $a = \lim_n x_{k_n}.$ Potom podle $d(x_n,y_n) < \frac{1}{n}$ je též 
	$a = \lim_n y_{k_n}.$ Podle $d'(f(x_n),f(y_n)) \geq \varepsilon$ nemůže být $f(a) = \lim_n f(x_{k_n})$ a zároveň $f(a) = \lim_n f(y_{k_n})$, 
	a tedy $f$ není ani spojité.
\end{proof}

\end{document}
