\documentclass[../main.tex]{subfiles}

\begin{document}

%%%%%%%%%%%%%%%%%%%%%%%%%%%%%%%%%%%%%%%%%%%%%%%%%%%%%%%%%%%%%%%%%%%%%%%%%%%%%%%%%%%%%%%%%%%%%%%%%%%%%%%%%
\section{Implicitní funkce}

\begin{example}
	Mějme $F(x,y) = x^2 + y^2 - 1$, neboli rovnici \( x^2 + y^2 = 1 \):
	\begin{itemize}
	    \item Pro některá $x_0$ jako například $x_0 < -1$ řešení neexistuje, o funkci $y(x)$ nemluvě.
	    \item Přestože řešení v nějakém okolí $x_0$ existuje, nemůžeme v nějakých situacích hovořit o funkci. Potřebujeme kolem řešení $(x_0, y_0)$ vymezit okolí jak $x_0$, tak $y_0$.
	    \item Máme také případy, jako ten, kdy $x_0 = 1$, kde je v okolí mnoho řešení, ale žádný (ani jednostranný) interval, kde by $y$ bylo jednoznačné.
	\end{itemize}
	V případě $F(x,y)$ už zádná další situace nenastane.
\end{example}

\begin{example}[Obecný]
	Mějme spojité reálné funkce $F_i(x_1, ... , x_m, y_1, ... , y_n)$ pro každé $i \in \{1, ..., n\}$
	v $n + m$
	proměnných. Určuje systém rovnic
	\[
		\begin{aligned}
			F_1(x_1, ... , x_m, y_1, ... , y_n) &= 0 \\
			\vdots \hspace{15mm} \vdots \hspace{15mm} \vdots \\
			F_n(x_1, ... , x_m, y_1, ... , y_n) &= 0
		\end{aligned}
\]
	v nějakém smyslu funkce
	\[ f_i \equiv y_i(x_1, ... , x_m) \]
	pro $i \in \{ 1, ... , n \}$? Pokud ano, jak a kde je určuje a jaké mají funkce vlastnosti?
\end{example}

\begin{intuition}
	3b1b má na svém YouTubu o úvodu do implicitních funkcí hezké video [\href{https://www.youtube.com/watch?v=qb40J4N1fa4}{odkaz}].
\end{intuition}

\subsection{Věty o implicitní funkci}

\begin{theorem}
	Buď $F(x,y)$ reálná funkce definovaná v nějakém okolí bodu $(x_0, y_0)$. Nechť má $F$ spojité parciální
	derivace do řádu $k \geq 1$ a nechť platí:
	\begin{align*}
	    F(x_0, y_0) &= 0\\
	    \left| \frac{\partial F(x_0,y_0)}{\partial y} \right| &\neq 0
	\end{align*}
	Potom $ \exists \delta > 0$ a $\Delta > 0$ takové, že
	$\forall x \in (x_0 - \delta , x_0 + \delta)\ \exists! y \in (y_0 - \Delta , y_0 + \Delta): F(x,y) = 0$.
	Dále, označíme-li toto jediné $y$ jako $y = f(x)$, potom získaná
	$f: (x_0 - \delta , x_0 + \delta ) \to \mathbb{R}$ má spojité derivace do řádu $k$.
\end{theorem}

\begin{proof}
	Buď třeba
	$$
	\frac{\partial F(x_0, y_0)}{\partial y} > 0
	$$
	Vzhledem ke spojitosti existují $\delta_1, \Delta > 0$ taková, že v
	obdélníku $\langle x_0 - \delta_1, x_0 + \delta_1 \rangle \times
	\langle y_0 - \Delta, y_0 + \Delta \rangle$ je $\frac{\partial
	F}{\partial y} (x, y)$ stále kladná. Tento obdélník je kompaktní, tedy
	na něm spojité funkce nabývají extrémů a tedy
	$\exists a, A > 0:$
	$$
	\frac{\partial F}{\partial y}(x, y) > a,~ \left |
	\frac{\partial F}{\partial x}(x, y) \right | < A
	$$
	Vezměme funkci $\varphi(y) = F(x_0, y)$. Ta má kladnou derivaci a tedy
	roste na intervalu $\langle y_0 - \Delta, y_0 + \Delta \rangle$.
	Jelikož $\varphi(y_0) = 0$, je $F(x_0, y_0 - \Delta) = \varphi(y_0 -
	\Delta) < 0$ a $F(x_0, y_0 + \Delta) > 0$. Jelikož funkce $F$ je
	spojitá (má spojité parciální derivace a tedy i totální diferenciál),
	existuje $\delta > 0$ takové, že $F(x, y_0 - \Delta) < 0$ a $F(x, y_0 +
	\Delta) > 0$ pro všechna $x \in (x_0 - \delta, x_0 + \delta)$. Pro
	taková $x$ položme $\varphi_x(y) = F(x, y)$. Máme vždy $\varphi_x'(y) >
	0$ na celém $\langle y_0 - \Delta, y_0 + \Delta \rangle$, takže je tam
	každá z těchto funkcí rostoucí a ovšem spojitá. Jelikož $\varphi_x(y_0
	- \Delta) < 0 < \varphi(y_0 + \Delta)$, nabývá $F(x, y) =
	\varphi_x(y)$, v intervalu $(y_0 - \Delta, y_0 + \Delta)$ nuly v právě
	jednom bodě $y$. Označme toto $y$ jako $f(x)$.

	Zatím však ani nevíme, zda je $f$ spojitá. Pojďme odvodit vlastnosti
	$f$.
	\begin{align*}
		0 &= F(x + h, f(x + h)) - F(x, f(x)) \\
		&= \frac{\partial F (x + \theta h, f(x) + \theta (f(x + h) -
		f(x)))}{\partial x} \cdot h \\
		&+ \frac{\partial F (x + \theta h, f(x) + \theta (f(x + h) -
		f(x)))}{\partial y} \cdot (f(x + h) - f(x))
	\end{align*}
	První rovnost platí, protože $F(t, f(t)) = 0$. Druhá rovnost platí z
	Lagrangeovy věty ve tvaru $f(\mathbf{x} + \mathbf{h}) - f(\mathbf{x}) =
	\sum_{j=1}^{n} \frac{\partial f(\mathbf{x} + \theta
	\mathbf{h})}{\partial x_j}h_j$, kde dosadíme $\mathbf{h} = (h, f(x + h)
	- f(x))$.

	Upravíme
	$$
	f(x + h) - f(x) = -h \cdot
	\frac{\frac{\partial F(...)}{\partial x}}
	{\frac{\partial F(...)}{\partial y}}
	$$
	Podle výše získaných nerovností s $a$ a $A$ tedy $|f(x + h) - f(x)| <
	|h| \cdot \frac{A}{a}$ a $f$ je tedy spojité v bodě $x$. Vrátíme se
	ještě jednou k $f(x + h) - f(x)$ a dostaneme
	$$
	\frac{f(x + h) - f(x)}{h} = - \left ( \frac{\partial F (x + \theta h,
	f(x) + \theta (f(x + h) - f(x)))}{\partial y} \right)^{-1} \cdot
	\frac{\partial F(..., ...)}{\partial x}
	$$
	Nyní již ale víme, že $f$ je spojitá, a spojitost $\frac{\partial
	F}{\partial x}$ a $\frac{\partial F}{\partial y}$ byla v předpokladech.
	Tedy má pravá strana limitu pro $h \rightarrow 0$ a dostáváme
	(dle věty o konvergenci můžeme dosadit $h = 0$)
	$$
	f'(x) = - \left ( \frac{\partial F(x, y)}{\partial y} \right )^{-1}
	\cdot \frac{\partial F(x, y)}{\partial x}
	$$
	Tuto formuli můžeme nyní derivovat tak dlouho, jak to existence PD na
	pravé straně dovolí.
\end{proof}

%%%%%%%%%%%%%%%%%%%%%%%%%%%%%%%%%%%%%%%%%%%%%%%%%%%%%%%%%%%%%%%%%%%%%%%%%%%%%%%%%%%%%%%%%%%%%%%%%%%%%%%%%
% Tohle tam psat nebudu, neni to uplne necessary a neukaze ti to nic uplne novyho
%\subsection{Definice (implicitní?) funkce a vlastnosti}
%\hspace{1.2mm}
%(5. str 7)
%\noindent

%%%%%%%%%%%%%%%%%%%%%%%%%%%%%%%%%%%%%%%%%%%%%%%%%%%%%%%%%%%%%%%%%%%%%%%%%%%%%%%%%%%%%%%%%%%%%%%%%%%%%%%%%

\begin{definition}[Jacobiho determinant]
	Pro konečnou posloupnost funkcí
	\[ \mathbf{F}(\mathbf{x}, \mathbf{y}) =
	(F_1(\mathbf{x}, y_1, ..., y_m), ... , F_m(\mathbf{x}, y_1, ..., y_m)) \]
	a pro $\mathbf{y} = (y_1, ... , y_m)$ se definuje Jacobiho determinant (Jakobián) jako
	\[ \frac{D(\mathbf{F})}{D(\mathbf{y})} =
	\det \left( \frac{\partial F_i}{\partial y_j} \right)_{i,j \in \{ 1, ... , m\}} \]
\end{definition}

\begin{intuition}
	Stejně jako determinant v lineární algebře určuje, jak daná matice transformuje prostor (natahuje vektory v daných směrech), tak Jacobiho matice určuje, jak vektorová funkce $\mathbf{f} = \left(f_1, \ldots, f_n\right)$ při transformaci oblasti $U \subseteq \mathrm{E}_n$ na $\mathbf{f}[U]$ natahuje nebo stlačuje objemy malých kousků oblasti $U$ okolo $\mathbf{x}$ v poměru (absolutní hodnoty) Jakobiánu.
\end{intuition}

\begin{remark}
	3b1b má o Jakobiánu na KhanAcademy super video [\href{https://www.khanacademy.org/math/multivariable-calculus/multivariable-derivatives/jacobian/v/the-jacobian-matrix}{odkaz}].
\end{remark}

\begin{theorem}
	Buďte $F_i(\mathbf{x}, y_1, ... , y_m)$ pro $i \in {1, ... , m}$ funkce $n+m$ proměnných se spojitými
	parciálními derivacemi do řádu $k \geq 1$. Buď \[ \mathbf{F}(\mathbf{x}^0, \mathbf{y}^0) = \mathbf{o} \]
	%a buď totální diferenciál v bodě $(\mathbf{x}^0, \mathbf{y}^0)$ 
	\[ \frac{D(\mathbf{F})}{D(\mathbf{y})}(\mathbf{x}^0, \mathbf{y}^0) \neq 0 \]
	Potom existují $\delta > 0$ a $\Delta > 0$ takové, že pro každé
	\[ \mathbf{x} \in (x_{1}^{0} - \delta, x_{1}^{0} + \delta) \times \cdot \cdot \cdot \times 
	(x_{n}^{0} - \delta, x_{n}^{0} + \delta)\]
	existuje právě jedno
	\[ \mathbf{y} \in (y_{1}^{0} - \Delta , y_{1}^{0} + \Delta) \times \cdot \cdot \cdot \times
	(y_{m}^{0} - \Delta , y_{m}^{0} + \Delta) \]
	takové, že
	\[ \mathbf{F}(\mathbf{x}, \mathbf{y}) = 0 \]
\end{theorem}

\begin{intuition}
	Místo důkazu si rozmysleme jednoduchý případ použití věty. Uvažujme
	dvojici rovnic
	\begin{align*}
		F_1(\mathbf{x}, y_1, y_2) &= 0, \\
		F_2(\mathbf{x}, y_1, y_2) &= 0 \\
	\end{align*}
	a pokusme se najít řešení $y_i = f_i(\mathbf{x}), i = 1, 2$, v nějakém
	okolí bodu $(\mathbf{x}^0, y_1^0, y_2^0)$. Aplikujme větu o jedné
	rovnici. Mysleme o druhé z rovnic jako o rovnici pro $y_2$; v nějakém
	okolí bodu $(\mathbf{x}^0, y_1^0, y_2^0)$ získáme $y_2$ jako funkci
	$\psi(\mathbf{x}, y_1)$. Substituujme ji do první rovnice, což dá
	$$
	G(\mathbf{x}, y_1) = F_1(\mathbf{x}, y_1, \psi(\mathbf{x}, y_1))
	$$
	a řešení $y_1 = f_1(\mathbf{x})$ v nějakém okolí bodu $(\mathbf{x}^0,
	y_1^0)$ může být substituováno do $\psi$ a získáme $y_2 =
	f_2(\mathbf{x}) = \psi(\mathbf{x}, f_1(\mathbf{x}))$.

	Co jsme vlastně předpokládali:
	\begin{itemize}
		\item Spojité parciální derivace funkcí $F_i$
		\item $\frac{\partial F_2}{\partial y_2} (\mathbf{x}^0, y_1^0,
			y_2^0) \neq 0$
		\item Konečně jsme potřebovali (použijte řetízkové pravidlo)
			$\frac{\partial G}{\partial y_1} (\mathbf{x}^0, y_1^0)
			= \frac{\partial F_1}{\partial y_1} + \frac{\partial
			F_1}{\partial y_2} \frac{\partial \psi}{\partial y_1}
			\neq 0$ (*)
	\end{itemize}
	Užijme formuli, kterou již máme (z důkazů jednoduché verze věty)
	$$
	\frac{\partial \psi}{\partial y_1} = - \left ( \frac{\partial
	F_2}{\partial y_2} \right )^{-1} \frac{\partial F_2}{\partial y_1}
	$$
	což transformuje (*) na
	$$
	\left ( \frac{\partial F_2}{\partial y_2} \right )^{-1} \left (
	\frac{\partial F_1}{\partial y_1} \frac{\partial F_2}{\partial y_2} -
	\frac{\partial F_1}{\partial y_2} \frac{\partial F_2}{\partial y_1}
	\right ) \neq 0
	$$
	to jest,
	$$
	\left (
	\frac{\partial F_1}{\partial y_1} \frac{\partial F_2}{\partial y_2} -
	\frac{\partial F_1}{\partial y_2} \frac{\partial F_2}{\partial y_1}
	\right ) \neq 0
	$$
	To je vzorec, který známe, totiž determinant. Takže jsme vlastně
	předpokládali, že
	$$
	\det \left ( \frac{\partial F_i}{\partial y_j} \right )_{i,j} \neq 0
	$$
	A tato podmínka již stačí: není-li determinant nula, je \emph{buď}
	$$
	\frac{\partial F_2}{\partial y_2} (\mathbf{x}^0, y_1^0, y_2^0) \neq 0
	$$
	\emph{a/nebo}
	$$
	\frac{\partial F_2}{\partial y_1} (\mathbf{x}^0, y_1^0, y_2^0) \neq 0
	$$
	Platí-li tedy to druhé, můžeme začít řešením rovnice $F_2(\mathbf{x},
	y_1, y_2) = 0$ pro $y_1$ místo $y_2$.
\end{intuition}

%%%%%%%%%%%%%%%%%%%%%%%%%%%%%%%%%%%%%%%%%%%%%%%%%%%%%%%%%%%%%%%%%%%%%%%%%%%%%%%%%%%%%%%%%%%%%%%%%%%%%%%%%
\end{document}
